\documentclass[letterpaper,12pt]{book}
%\usepackage[a4paper,includeall,bindingoffset=0cm,margin=2cm,marginparsep=0cm,marginparwidth=0cm]{geometry}
\usepackage[top=1in, left=0.9in, right=1.25in, bottom=1in]{geometry}
\usepackage{bachelorstitlepageUNAM} %portada
\usepackage[utf8]{inputenc}
%%%%%%%%%%%%%%%%%%%%%%%%%%%%%


\usepackage[T1]{fontenc}
\usepackage[utf8]{inputenc}
\usepackage[spanish,es-nodecimaldot,es-tabla]{babel}
\usepackage{graphicx}
\usepackage{tikz} 
\usepackage{tocloft}
\graphicspath{{./figs/}}
\usepackage{setspace}
\spacing{1.5}
%\usepackage{apalike}
\usepackage{amsmath}
\usepackage{blindtext}
\usepackage[backend=biber,style=apa]{biblatex}
\bibliography{Bibliografia.bib}
%\bibliographystyle{apalike}
\usepackage{hyperref} 
\usepackage{multirow}%permite la modificacion de las tablas
%\usepackage[table,xcdraw]{xcolor}
\usepackage{colortbl}
\usepackage{float}
\usepackage{amssymb, amsmath}
\decimalpoint
\usepackage{listings}
\usepackage{amsfonts}
\usepackage{bbm}
\usepackage{bm}
\usepackage{mathtools}
%\usepackage{caption}
%\usepackage{subcaption}
\usepackage{matlab-prettifier}
\usepackage{amsthm}
\usepackage{titlesec}
\newtheorem{theorem}{Teorema}
\newtheorem{proposition}[theorem]{Proposición}
\newtheorem{lemma}[theorem]{Lema}
\newtheorem{corollary}[theorem]{Corolario}
\titleformat{\chapter} % Config titlesec
	{\Large\bfseries}	% 
	{\huge \thechapter}	% 
	{20pt}				% 
	{\huge}				% 
	
\usepackage[titletoc]{appendix} % Añade palabra «Apéndice» en Índice y define el ambiente appendices
\usepackage{caption}
\usepackage{subfiles} % Manejo de varios archivos, por ejemplo: capítulos escritos en documentos por separado.
\usepackage{subcaption}

%\usepackage[round]{natbib}

\renewcommand\cftsecpresnum{\S}
\renewcommand\cftsubsecpresnum{\S}   
%%%%-------------Aquí empieza el proyecto-----------------%%%%%%

\begin{document}
%%-------------Portada-----------------%%%%
    \begin{titlepage}
        \thispagestyle{empty}
        \begin{minipage}[c][0.17\textheight][c]{0.25\textwidth}
            \begin{center}
                \includegraphics[width=3.5cm, height=3.5cm]{Escudo-UNAM.pdf}
            \end{center}
        \end{minipage}
        \begin{minipage}[c][0.195\textheight][t]{0.75\textwidth}
            \begin{center}
                \vspace{0.3cm}
                \textsc{\large Universidad Nacional Aut\'onoma de M\'exico}\\[0.5cm]
                \vspace{0.3cm}
                \hrule height2.5pt
                \vspace{.2cm}
                \hrule height1pt
                \vspace{.8cm}
                \textsc{Facultad de Ciencias}\\[0.7cm] %
            \end{center}
        \end{minipage}

        \begin{minipage}[c][0.81\textheight][t]{0.25\textwidth}
            \vspace*{5mm}
            \begin{center}
                \hskip2.0mm
                \vrule width1pt height13cm 
                \vspace{5mm}
                \hskip2pt
                \vrule width2.5pt height13cm
                \hskip2mm
                \vrule width1pt height13cm \\
                \vspace{5mm}
                \includegraphics[height=4.0cm]{Escudo-facultad-ciencias.png}
            \end{center}
        \end{minipage}
        \begin{minipage}[c][0.81\textheight][t]{0.75\textwidth}
            \begin{center}
                \vspace{1cm}

                {\large\scshape Análisis de la cienciometría en biociencias}\\[.2in]

                \vspace{2cm}            

                \textsc{\LARGE T\hspace{.8cm}E\hspace{.8cm}S\hspace{.8cm}I\hspace{.8cm}S\hspace{.8cm}}\\[0.5cm]
                \textsc{\large que para obtener el t\'itulo de:}\\[0.5cm]
                \textsc{\large Matemáticas Aplicadas}\\[0.5cm]
                \textsc{\large presenta:}\\[0.5cm]
                \textsc{\large {Minerva María Romero Pérez}}\\[2cm]          

                \vspace{0.5cm}

                {\large\scshape Directora de tesis:\\[0.3cm] {Dra. Layla Michán Aguirre }}\\[.2in]

                \vspace{0.5cm}

                \large{Ciudad Universitaria,Ciudad de México}{ }{2024}
            \end{center}
        \end{minipage}
    \end{titlepage}
%---------------------------------
\frontmatter
%%-------------Taba de contenido-----------------%%%%
\tableofcontents
\mainmatter % si pones esto despues de una parte, el indice se pondra despues de esta parte y los temas de la tesis vendaran despues 

%%-------------Se manda a llamar a cada sección desde la carpeta de capitulos -----------------%%%%
\chapter{Introducción}
\subfile{capitulos/introduccion}

\chapter{Material y método}
\subfile{capitulos/material_y_metodo}

\chapter{Resultados}
\subfile{capitulos/resultados}

\chapter{Discusión}
\subfile{capitulos/discusion}


%\backmatter
%%-------------Pone la bibliografía -----------------%%%%

\printbibliography
\backmatter
\end{document}